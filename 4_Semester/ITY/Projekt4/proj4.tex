% ITY Projekt 4
% Vypracoval: Michal Pyšík (login: xpysik00)

\documentclass[a4paper, 11pt]{article}

\usepackage[utf8]{inputenc}
\usepackage[czech]{babel}
\usepackage[left=2cm, top=3cm, text={17cm, 24cm}]{geometry}
\usepackage{times}
\usepackage[unicode]{hyperref}


\begin{document}

\begin{titlepage}
\begin{center}
    \textsc{\Huge Vysoké učení technické v~Brně\\ \huge Fakulta informačních technologií\\}
    \vspace{\stretch{0.382}}
    \LARGE Typografie a publikování\,--\,4. projekt\\
    \Huge Bibliografické citace
    \vspace{\stretch{0.618}}
\end{center}
\Large{\today \hfill Michal Pyšík}
\bigskip
\end{titlepage}


\section*{Matematická analýza}

\subsection*{Úvod}
Matematická analýza je jednou z~nejzákladnějších matematických disciplín. Nalezneme v ní pojmy jako jsou například limity, 
derivace, integrály a~nekonečné řady \cite{Hirschman_2014}. Obecně lze však říci, že zkoumá především změny, respektive rychlost změny
(anglicky rate of change) \cite{Cook_2017}.
Ve světě je tato disciplína známá pod názvem \textbf{calculus}, od roku 2000 se tento název začal prosazovat také do češtiny
\cite{Cerny_2002}.


\subsection*{Historie}
V dnešní době je obecně uznáváno, že základy matematické analýzy objevili v~pozdním 17. století Isaac Newton
a~Gottfried Wilhelm Leibniz, a~to nezávisle na sobě. Více o~její historii se lze dočíst například v~\cite{Rosenthal_1951}.


\subsection*{Limity}
Limita popisuje chování nějaké funkce v okolí určitého bodu, díky čemu můžeme například definovat spojitost funkce. Limita funkce nám pomůže
pochopit chování funkce i~v~místech, ve kterých není vůbec definovaná \cite{matematika_cz}.


\subsection*{Diferenciální počet}
Právě diferenciální počet zkoumá již dříve zmíněné rychlosti změn.
Jedním ze základních faktů je, že první derivace zadané funkce v daném bodě nám udává směrnici tečny funkce procházející tímto bodem.
Mezi nejdůležitější praktické aplikace patří například nalezení lokálních extrémů dané funkce za účelem řešení optimalizačních problémů
\cite{Britannica_2011_dif}. S~takovými problémy se můžeme setkat nejen v~technických oborech, ale také například v~ekonomice,
viz \cite{Jankensgard_2020}.

\subsection*{Integrální počet}
Dalo by se říci, že integrace je proces symetricky opačný derivacím, proto v~anglické literatuře můžeme
často narazit na pojem \textbf{antiderivative} \cite{Britannica_2016_int}. Z~geometrického hlediska se jedná o~výpočet plochy
pod grafem dané funkce,
k~čemuž slouží určitý (Riemannův) integrál, viz \cite{Butek_2005}. Nemusíme však zůstavat ve dvoudimenzionálním prostoru,
tento koncept lze zobecnit pro funkce více proměnných, poté se bavíme o~vícenásobných integrálech~\cite{Valesova_2016}.


\newpage
\bibliographystyle{czechiso}
\renewcommand{\refname}{Literatura}
\bibliography{citace}

\end{document}
