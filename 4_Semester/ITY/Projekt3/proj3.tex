% ITY Projekt 3
% Vypracoval: Michal Pyšík (login: xpysik00)

\documentclass[a4paper, 11pt]{article}

\usepackage[utf8]{inputenc}
\usepackage[czech]{babel}
\usepackage[left=2cm, top=3cm, text={17cm, 24cm}]{geometry}
\usepackage{times}
\usepackage[ruled, czech, linesnumbered, noline, longend]{algorithm2e}
\usepackage{graphics}
\usepackage{multirow}
\usepackage[hidelinks, unicode]{hyperref}
\urlstyle{same}
\usepackage{pdflscape}
\usepackage{pict2e}


\begin{document}

\begin{titlepage}
\begin{center}
    \textsc{\Huge Vysoké učení technické v~Brně\\ \huge Fakulta informačních technologií\\}
    \vspace{\stretch{0.382}}
    \LARGE Typografie a publikování\,--\,3. projekt\\
    \Huge Tabulky a obrázky
    \vspace{\stretch{0.618}}
\end{center}
\Large{\today \hfill Michal Pyšík}
\bigskip
\end{titlepage}


\section{Úvodní strana}
Název práce umístěte do zlatého řezu a~nezapomeňte uvést dnešní datum a~vaše jméno a~příjmení.


\section{Tabulky}
Pro sázení tabulek můžeme použít buď prostředí\verb| tabbing| nebo prostředí\verb| tabular|.

\subsection{Prostředí \texttt{tabbing}}
Při použítí\verb| tabbing |vypadá tabulka následovně:

\begin{tabbing}
    Vodní melouny \quad \= 35,-- \quad \= \kill
    \textbf{Ovoce} \quad \> \textbf{Cena} \> \textbf{Množství} \\
    Jablka \> 25,90 \> 3 kg \\
    Hrušky \> 27,40 \> 2,5 kg \\
    Vodní melouny \> 35,-- \> 1 kus \\
\end{tabbing}
Toto prostředí se dá také využít pro sázení algoritmů, ovšem vhodnější je použít prostředí\verb| algorithm |
nebo \verb|algorithm2e |(viz sekce \ref{sekce_3}).

\subsection{Prostředí \texttt{tabular}}
Další možností, jak vytvořit tabulku, je použít prostředí\verb| tabular|. Tabulky pak budou vypadat
takto\footnote{Kdyby byl problém s \texttt{cline}, zkuste se podívat třeba sem: \url{http://www.abclinuxu.cz/tex/poradna/show/325037}.}:

\bigskip

\begin{table}[h]
\catcode`\-=12
\centering
\begin{tabular}{|l|r|r|}
    \hline
    & \multicolumn{2}{|c|}{\textbf{Cena}} \\
    \cline{2-3}
    \textbf{Měna} & \textbf{nákup} & \textbf{prodej} \\
    \hline
    EUR & 25,227 & 26,943 \\
    GBP & 29,368 & 31,492 \\
    USD & 21,260 & 22,661 \\
    \hline
\end{tabular}
\caption{Tabulka kurzů k~dnešnímu dni}
\label{tabulka_1}
\end{table}

\bigskip

\begin{table}[h]
\catcode`\-=12
\centering
\begin{tabular}{|c|c|}
    \hline
    $A$ & $\neg A$ \\
    \hline
    \textbf{P} & N \\
    \hline
    \textbf{O} & O \\
    \hline
    \textbf{X} & X \\
    \hline
    \textbf{N} & P \\
    \hline
\end{tabular}
%%%%%%%%%%%%%%%%%%%%%%%%%%%%%%%%%%%%%%%%%%%%%%%%%%%%%%
\begin{tabular}{|c|c|c|c|c|c|}
    \hline
    \multicolumn{2}{|c|}{\multirow{2}{*}{$A \wedge B$}} & \multicolumn{4}{|c|}{$B$} \\
    \cline{3-6}
    \multicolumn{2}{|c|}{} & \textbf{P} & \textbf{O} & \textbf{X} & \textbf{N} \\
    \hline
    \multirow{4}{*}{$A$} & \textbf{P} & P & O & X & N \\
    \cline{2-6}
    & \textbf{O} & O & O & N & N \\
    \cline{2-6}
    & \textbf{X} & X & N & X & N \\
    \cline{2-6}
    & \textbf{N} & N & N & N & N \\
    \hline
\end{tabular}
%%%%%%%%%%%%%%%%%%%%%%%%%%%%%%%%%%%%%%%%%%%%%%%%%%%%%%
\begin{tabular}{|c|c|c|c|c|c|}
    \hline
    \multicolumn{2}{|c|}{\multirow{2}{*}{$A \vee B$}} & \multicolumn{4}{|c|}{$B$} \\
    \cline{3-6}
    \multicolumn{2}{|c|}{} & \textbf{P} & \textbf{O} & \textbf{X} & \textbf{N} \\
    \hline
    \multirow{4}{*}{$A$} & \textbf{P} & P & P & P & P \\
    \cline{2-6}
    & \textbf{O} & P & O & P & O \\
    \cline{2-6}
    & \textbf{X} & P & P & X & X \\
    \cline{2-6}
    & \textbf{N} & P & O & X & N \\
    \hline
\end{tabular}
%%%%%%%%%%%%%%%%%%%%%%%%%%%%%%%%%%%%%%%%%%%%%%%%%%%%%%
\begin{tabular}{|c|c|c|c|c|c|}
    \hline
    \multicolumn{2}{|c|}{\multirow{2}{*}{$A \vee B$}} & \multicolumn{4}{|c|}{$B$} \\
    \cline{3-6}
    \multicolumn{2}{|c|}{} & \textbf{P} & \textbf{O} & \textbf{X} & \textbf{N} \\
    \hline
    \multirow{4}{*}{$A$} & \textbf{P} & P & P & P & P \\
    \cline{2-6}
    & \textbf{O} & P & O & P & O \\
    \cline{2-6}
    & \textbf{X} & P & P & X & X \\
    \cline{2-6}
    & \textbf{N} & P & O & X & N \\
    \hline
\end{tabular}
\caption{Protože Kleeneho trojhodnotová logika už je \uv{zastaralá}, uvádíme si zde příklad čtyřhodnotové logiky}
\label{tabulka_2}
\end{table}
\bigskip


\pagebreak
\section{Algoritmy}
\label{sekce_3}
Pokud budeme chtít vysázet algoritmus, můžeme použít prostředí\texttt{ algorithm}\footnote{Pro nápovědu jak zacházet
s~prostředím\texttt{ algorithm, }můžeme zkusit tuhle stránku:\\ \url{http://ftp.cstug.cz/pub/tex/CTAN/macros/latex/contrib/algorithms/algorithms.pdf}.}
nebo \texttt{ algorithm2e}\footnote{Pro\texttt{ algorithm2e }zase tuhle: \url{http://ftp.cstug.cz/pub/tex/CTAN/macros/latex/contrib/algorithm2e/doc/algorithm2e.pdf}.}
Příklad použití prostředí\verb| algorithm2e |viz Algoritmus \ref{algoritmus_1}.
\bigskip

\begin{algorithm}[h]
    \label{algoritmus_1}
    \caption{\textsc{FastSLAM}}
    \SetInd{1em}{1em}
    \SetNlSty{}{}{:}
    \SetNlSkip{-1.2em}
    \SetAlgoNlRelativeSize{-1}
    
    \KwIn{$ (X_{t-1}, u_t, z_t) $}
	\KwOut{$ X_t $}
	\BlankLine
	\Indpp \Indp
	$ \overline{X_t} = X_t = 0 $ \\
	\For{$ k = 1 $ \textnormal{to} $M$}
	    { $ x_t^{[k]} = sample\_motion\_model(u_t, x_{t-1}^{[k]}) $ \\
	      $ \omega_t^{[k]} = measurement\_model(z_t, x_t^{[k]}, m_{t-1}) $ \\
	      $ m_t^{[k]} = updated\_occupancy\_grid(z_t, x_t^{[k]}, m_{t-1}^{[k]}) $ \\
	      $ \overline{X_t} = \overline{X_t} + \langle x_x^{[k]}, \omega_t^{[m]} \rangle $ }
    \For{$ k = 1 $ \textnormal{to} $M$}
        { draw $i$ with propability $ \approx \omega_t^{[i]} $ \\
          add $ \langle x_x^{[k]}, m_t^{[k]} \rangle $ to $X_t$ }
    \Return{$X_t$}
\end{algorithm}


\section{Obrázky}
Do našich článků můžeme samozřejmě vkládat obrázky. Pokud je obrázkem fotografie,
můžeme klidně použít bitmapový soubor. Pokud by to ale mělo být nějaké schéma nebo něco podobného,
je dobrým zvykem takový obrázek vytvořit vektorově.

\begin{figure}[h]
    \centering
    \scalebox{0.4}
        { \includegraphics{etiopan.eps} \reflectbox{\includegraphics{etiopan.eps}} }
    \caption{Malý Etiopánek a~jeho bratříček}
    \label{obrazek_1}
\end{figure}
\bigskip
\pagebreak

Rozdíl mezi vektorovým \dots
\begin{figure}[h]
    \centering
    \scalebox{0.4}
        { \includegraphics{oniisan.eps} }
    \caption{Vektorový obrázek}
    \label{obrazek_2}
\end{figure}
\bigskip

\noindent{\dots\,a~bitmapovým obrázkem}
\begin{figure}[h]
    \centering
    \scalebox{0.6}
        { \includegraphics{oniisan2.eps} }
    \caption{Bitmapový obrázek}
    \label{obrazek_3}
\end{figure}
\bigskip

\noindent{se projeví například při zvětšení.}

Odkazy (nejen ty) na obrázky \ref{obrazek_1}, \ref{obrazek_2} a~\ref{obrazek_3},
na tabulky \ref{tabulka_1} a~\ref{tabulka_2} a také na algoritmus \ref{algoritmus_1}
jsou udělány pomocí křížových odkazů. Pak je ovšem nutné zdrojový soubor přeložit dvakrát.

Vektorové obrázky lze vytvořit přímo v \LaTeX u, například pomocí prostředí\verb| picture.|

\begin{landscape}
\begin{figure}[h]
\centering
\setlength{\unitlength}{1mm}
\begin{picture}(200, 120)
    \linethickness{0.5mm}
    \put(0,0){\framebox(200, 120){}}
    \linethickness{1mm}
    \put(5, 10){\line(1,0){190}}
    
    \linethickness{0.4mm}
    \put(20, 10){\line(0,1){80}}
    \put(130, 42){\line(0,1){48}}
    \put(130, 42){\line(-1,0){10}}
    \put(130, 42){\line(-1,0){10}}
    \put(120, 45){\line(0,-1){3}}
    \put(120, 45){\line(-1,0){25}}
    \put(95, 45){\line(0,-1){35}}
    
    \linethickness{0.6mm}
    \put(19,90){\line(1,0){112}}
    
    \linethickness{0.3mm}
    \put(122,10){\line(0,1){32}}
    \put(128,10){\line(0,1){32}}
    \put(20, 13){\line(1,0){75}}
    \put(37, 13){\line(0,1){30}}
    \put(37, 43){\line(1,0){40}}
    \put(77, 13){\line(0,1){30}}
    \put(45, 60){\framebox(69,22){}}
    \put(170, 95){\circle{30}}
    
    \linethickness{0.2mm}
    \put(56, 13){\line(0,1){30}}
    \put(58, 13){\line(0,1){30}}
    \put(67, 60){\line(0,1){22}}
    \put(69, 60){\line(0,1){22}}
    \put(89, 60){\line(0,1){22}}
    \put(91, 60){\line(0,1){22}}
    
    \put(167, 100){\circle{4}}
    \put(165, 90){\circle{7}}
    \put(177, 93){\circle{5}}
\end{picture}
\caption{Vektorový obrázek mého domu při pohledu ze zahrady, společně s~Měsícem}
\end{figure}
\end{landscape}


\end{document}
