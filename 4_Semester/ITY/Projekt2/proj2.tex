\documentclass[twocolumn, a4paper, 11pt]{article}

% ITY Projekt 2
% Autor: Michal Pyšík (login: xpysik00)

\usepackage[utf8]{inputenc}
\usepackage[IL2]{fontenc}
\usepackage[czech]{babel}
\usepackage[left=1.5cm, top=2.5cm, text={18cm, 25cm}]{geometry}
\usepackage{times}
\usepackage{amsmath, amssymb, amsthm}
\usepackage{enumitem}
\usepackage[unicode, hidelinks]{hyperref}

\theoremstyle{definition}
\newtheorem{definition}{Definice}

\theoremstyle{plain}
\newtheorem{sentence}{Věta}


\begin{document}

\begin{titlepage}

\begin{center}
    \textsc{\Huge Fakulta informačních technologií\\Vysoké učení technické v~Brně\\}
    \vspace{\stretch{0.382}}
    \LARGE{Typografie a publikování -- 2. projekt\\Sazba dokumentů a matematických výrazů}
    \vspace{\stretch{0.618}}
\end{center}

\Large{\the\year\hfill Michal Pyšík (xpysik00)}

\end{titlepage}


\section*{Úvod}
V~této úloze si vyzkoušíme sazbu titulní strany, matematických vzorců, prostředí a~dalších textových struktur obvyklých pro technicky zaměřené texty (například rovnice~(\ref{equation_1})
nebo Definice \ref{definice_1} na straně \pageref{definice_1}). Rovněž si vyzkoušíme používání odkazů \verb|\ref| a~\verb|\pageref|.

Na titulní straně je využito sázení nadpisu podle optického středu s~využitím zlatého řezu. Tento postup byl
probírán na přednášce. Dále je použito odřádkování se
zadanou relativní velikostí 0.4 em a~0.3 em.

V případě, že budete potřebovat vyjádřit matematickou
konstrukci nebo symbol a nebude se Vám dařit jej nalézt
v samotném \LaTeX u, doporučuji prostudovat možnosti balíku maker \AmS -\LaTeX.


\section{Matematický text}
Nejprve se podíváme na sázení matematických symbolů
a~výrazů v~plynulém textu včetně sazby definic a~vět s~využitím balíku \verb|amsthm|.
Rovněž použijeme poznámku pod čarou s~použitím příkazu \verb|\footnote|.
Někdy je vhodné použít konstrukci \verb|\mbox{}|, která říká, že text nemá být zalomen.

\begin{definition}
    \label{definice_1}
    Rozšířený zásobníkový automat \emph{(RZA) je definován jako sedmice tvaru} $ A = (Q, \Sigma, \Gamma, \delta, q_0, Z_0, F) $,
    \emph{kde:}
    \renewcommand\labelitemi{$\hbox{\small$\bullet$}$} % nastaveni vetsich itemize kolecek
    \begin{itemize}
        \item $Q$ \emph{je konečná množina} vnitřních řídících stavů,
        \item $\Sigma$ \emph{je konečná} vstupní abeceda,
        \item $\Gamma$ \emph{je konečná} zásobníková abeceda,
        \item $\delta$ \emph{je} přechodová funkce
            $ Q \! \times \! ( \Sigma \! \cup \! \{ \epsilon \} ) \! \times \! {\Gamma}^* \rightarrow 2^{Q \times {\Gamma}^*} $,
        \item \emph{$ q_0 \in Q $ je} počáteční stav, \emph{$ Z_0 \in \Gamma $ je} startovací symbol zásobníku
            \emph{a}~$ F \subseteq Q $ \emph{je množina} koncových stavů.
    \end{itemize}
    
    Nechť $ P \, = \, ( Q, \Sigma, \Gamma, \delta, q_0, Z_0, F ) $ je rozšířený zásobníkový automat.
    \emph{Konfigurací} nazveme trojici $ ( q, w, \alpha ) \in Q \times {\Sigma}^* \times {\Gamma}^* $,
    kde $q$ je aktuální stav vnitřního řízení,\linebreak $w$ je dosud  nezpracovaná část vstupního řetězce
    a~$ \alpha = Z_{i_1} Z_{i_2} \dots Z_{i_k} $ je obsah zásobníku\footnote{$ Z_{i_1} $ je vrchol zásobníku}.
\end{definition}

\subsection{Podsekce obsahující větu a odkaz}
\begin{definition}
    \label{definice_2}
    Řetězec $w$ nad abecedou $\Sigma$ je přijat RZA\linebreak $A$ \emph{jestliže} $ ( q_0, w, Z_0 )
    \overset{*}{\underset{A}{\vdash}} ( q_F, \epsilon, \gamma ) $ \emph{pro nějaké} $ \gamma\!\in {\Gamma}^* $ \emph{a} $ q_F \in F $.
    \emph{Množinu} $ L(A) = \{ w \mid w $ \emph{je přijat RZA A} $ \} \subseteq {\Sigma}^* $~\emph{nazýváme} jazyk přijímaný RZA \emph{A}.
\end{definition}

Nyní si vyzkoušíme sazbu vět a~důkazů opět s~použitím balíku \verb|amsthm|.
\begin{sentence}
    \label{veta_1}
    Třída jazyků, které jsou přijímány ZA, odpovídá \textnormal{bezkontextovým jazykům.}
\end{sentence}

\begin{proof}
    V~důkaze vyjdeme z~definice \ref{definice_1} a~\ref{definice_2}.
\end{proof}

\section{Rovnice a odkazy}
Složitější matematické formulace sázíme mimo plynulý text.
Lze umístit několik výrazů na jeden řádek,
ale pak je třeba tyto vhodně oddělit, například příkazem \verb|\quad|.

$$
    \sqrt[i]{x_i^3}
    \quad
    \textnormal{kde } x_i \textnormal{ je } i \textnormal{-té sudé číslo splňující}
    \quad
    x_i^{x_i^{i^2}+2} \leq y_i^{x_i^4}
$$

V~rovnici (\ref{equation_1}) jsou využity tři typy závorek s~různou explicitně definovanou velikostí.

\begin{eqnarray}
    \label{equation_1}
    x & = & \bigg[ \Big\{ \big[ a + b \big] * c \Big\}^d \oplus 2 \bigg]^{3/2} \\
    y & = & \lim_{x \to \infty} \frac{ \frac{1}{\log_{10}x} }{ sin^2 x + cos^2 x } \nonumber
\end{eqnarray}

V této větě vidíme, jak vypadá implicitní vysázení limity $ \lim_{n \to \infty} f(n) $ v normálním odstavci textu.
Podobně je to i~s~dalšími symboly jako $ \prod^n_{i = 1} 2^i $ či $ \bigcap_{A \in \mathcal{B}} A $.
V~případě vzorců $ \lim\limits_{n \to \infty} f(n) $ a $ \prod\limits^n_{i = 1} 2^i $ jsme si vynutili méně
úspornou sazbu příkazem \verb|\limits|.

\begin{equation}
    \label{equation_2}
    \int_b^a g(x) \, dx \;\; = \;\; - \int\limits_a^b f(x) \, dx
\end{equation}

\section{Matice}
Pro sázení matic se velmi často používá prostření \verb|array| a~závorky (\verb|\left|, \verb|\right|).

$$
    \left(
    \begin{array}{ccc}
        a - b & \widehat{\xi + \omega} & \pi \\
        \Vec{\mathbf{a}} & \overleftrightarrow{AC} & \hat{\beta}
    \end{array}
    \right)
    = 1 \Longleftrightarrow \mathcal{Q} = \mathbb{R}
$$

$$
    \mathbf{A} =
    \left\|
    \begin{array}{cccc}
        a_{11} & a_{12} & \ldots & a_{1n} \\
         a_{21} & a_{22} & \ldots & a_{2n} \\
          \vdots & \vdots & \ddots & \vdots \\
           a_{m1} & a_{m2} & \ldots & a_{mn}
    \end{array}
    \right\|
    =
    \left|
    \begin{array}{cc}
        t & u \\
        v & w
    \end{array}
    \right|
    = tw - uv
$$

Prostředí \verb|array| lze úspěšně využít i~jinde.

$$
    \binom{n}{k} =
    \left\{
    \begin{array}{cl}
        0 & \textnormal{pro } k < 0 \textnormal{ nebo } k > n  \\
        \frac{n!}{k!(n-k)!} & \textnormal{pro } 0 \leq k \leq n \textnormal{.}
    \end{array}
    \right.
$$







\end{document}
